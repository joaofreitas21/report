\documentclass[a4paper,12pt,twoside]{book}
\usepackage{lesi/lesi}

\title{Atualização de Aplicação Web de Gestão de Processos de Formação}
\author{João Carlos Peixoto Freitas}
\nrAluno{16968}
\regimeDiurno         % ou \regimePosLaboral
\LESI
\date{2020/2021}

%% Caso tenham mais que um orientador, colocar
%% \orientador{Nome professor \AND Nome professor}
\orientador{Joaquim G. P. Silva}

%% comentar estas três linhas para projectos
\empresa{NKA - New Knowledge Advice, Lda}
\enderecoEmpresa{Travessa José Cruz, nº5, 4715-343 Braga, Portugal}
\supervisor{Engenheiro João Magalhães}

% comentar se não for para usar glossários
\makeglossaries




\begin{document}
\frontmatter
\maketitle  % print the title


\begin{resumo}
  %Resumo do trabalho realizado. Deve ser sucinto, e cobrir todo o relatório: uma introdução ao problema que se pretendeu resolver, um pequeno resumo da abordagem realizada, e algumas conclusões do trabalho atingido.
  %Poderão ser criados vários parágrafos, até para que cada um corresponda às três fases de introdução, desenvolvimento e conclusão.
  %Não é relevante colocar no resumo o local de estágio ou a referência ao curso.

\par A fim de manter o negócio competitivo no mercado, é preciso apostar em inovação. Por meio dessa iniciativa, a empresa consegue criar soluções eficientes, minimizar o impacto de eventualidades, antever situações adversas e gerar alternativas que proporcionem experiências positivas aos clientes.
\par Uma empresa que acredita que o seu produto já é bom o suficiente ao ponto de não precisar de inovação está destinada ao fracasso. Adiar ao máximo a atualização tecnológica, ou somente adotar medidas quando o mercado exige pode resultar numa perda de competitividade.
\par Tendo isto em consideração, com este trabalho foram atualizadas as tecnologias de um produto de forma a manter a sua operabilidade na tecnologia mais atual possível. Integraram-se no NkaAcademies, um produto desenvolvido pela NKA - \textit{New Knowledge Advice}, atualizações ao nível de linguagem da programação e das bibliotecas utilizadas de forma a tornar o produto o mais atual possível em termos tecnológicos. Foram ainda desenvolvidas novas funcionalidades para adicionar futuramente ao produto.
\par Por fim, foi feita uma análise ao trabalho realizado, apresentando soluções desenvolvidas e tirando algumas conclusões sobre o projeto e o estágio.
  %-- bibliografia eveo.com.br !!!

  %impossivel acompanhar todas as atualizações - e isso nem sempre é financeiramente vantajoso
\end{resumo}

\begin{abstract}

\par In order to keep the business competitive in the market, it is necessary to invest in innovation. With that in mind, the company is able to create efficient solutions, minimize the impact of eventualities, anticipate adverse situations and create alternatives that provide positive experiences for the costumer.
\par A company that believes its product is good enough, that it doesn't need innovation is doomed to failure. Postponing the technological update as much as possible, or only adopting measures when the market demands it, can result in a loss of competitiveness.
\par Keeping this in mind, with this work the technlogies of a product were updated in order to maintain its operability in the most current technology possible. In addition to NkaAcademies, a product developed by NKA - \textit{New Knowledge Advice}, updates to the programming language and libraries used were made in order to make the product as up-to-date as possible in terms of technology, and the development of new functionalities to be added to the product in the future.
\par Finally, an analysis was made of the work accomplished, presenting developed solutions and some conclusions were taken about the project and the internship.
\end{abstract}

%% Comment the following part if you are not acknowledging anybody
\begin{agradecimentos}
  %% [A secção de agradecimentos é a parte pessoal do documento, e o único sítio onde o aluno pode escrever de forma menos formal, usando o tipo de linguagem que lhe parecer adequado para as pessoas a quem agradece.] %%
\par O desempenho no desenvolvimento deste projecto de estágio não teria sido o mesmo sem a ajuda e apoio de algumas pessoas, às quais gostaria de aqui expressar o meu reconhecimento.
\par Assim, gostaria de começar por agradecer à empresa NKA - New Knowledge Advice pela oportunidade de realizar estágio numa época complicada que vivemos e por toda a ajuda ao longo do estágio. Um obrigado em especial ao meu orientador na empresa, Eng. João Magalhães, pelo acompanhamento ao longo do meu trabalho e pela disposição de transmitir valores profissionais. Aproveito também para agradecer à restante equipa da NKA pela simpatia e familiaridade durante este percurso.
\par Gostaria de agradecer também ao meu orientador de estágio, o professor Joaquim Silva pela orientação, disponibilidade e conselhos dados ao longo desta etapa.
\par Por fim, agradeço à minha família e amigos por todo o apoio pois é também graças a eles que consegui acabar esta etapa da minha vida.
\end{agradecimentos}


\tableofcontents

% comentar se nao tiver figuraIs
\listoffigures

% comentar se nao tiver tabelas
\listoftables

% comentar se nao se quiser lista de listagens
\lstlistoflistings

% Commentar proximas duas linhas se nao for para usar acronimos


\newacronym{php}{PHP}{HyperText Preprocessor }

\newacronym{sql}{SQL}{Structured Query Language}

\newacronym{xampp}{XAMPP}{Cross-Plataform(X), Apache(A), MariaDB(M), PHP (P) e Perl (P)}

\newacronym{moscow}{MoSCoW}{Must Have, Should Have, Could Have, Won't Have}

\printglossary[type=\acronymtype,title={Siglas \& Acrónimos}]

% Commentar proximas duas linhas se nao for para usar glossarios

\newglossaryentry{moscoww}
{
   name=MoSCoW,
   description={Método de priorização de funcionalidades ou requisitos }
}


\newglossaryentry{stemmer}
{
    name=stemmer,
    description={Ferramenta capaz de reduzir uma palavra à sua raiz. Por exemplo, para a palavra ``correria'', a sua raiz seria ``corre''. }
 }

\printglossary


%\printglossary


\mainmatter


\chapter{Introdução}

%[A introdução deve, tal como o próprio nome indica, introduzir o tema do trabalho. Não deve haver pressa em falar da empresa onde foi realizado o estágio ou o curso a que se refere o trabalho. Deve fazer-se uma introdução à área, Os Sistemas Informáticos ou as Ciências da Computação são áreas bastante grandes, pelo que não se deve supor que o leitor está a par das necessidades ou das tecnologias usadas em determinada área. No entanto, não devem ser explicados conceitos básicos, que qualquer licenciado numa engenharia de sistemas informáticos ou em ciências da computação tenham obrigação de conhecer.

Acompanhar as novas tecnologias é fundamental para um negócio. É preciso investir em inovação para manter um negócio no mercado competitivo. A tecnologia, aliada a uma boa atualização, consegue proporcionar soluções eficientes às empresas.

% Algo como o resumo é valido para a introdução ???
-- por terminar -- acabaram as ideias !!

\section{ Objetivos }

% - [Numa pequena secção da introdução liste, cuidadosamente, os objetivos do trabalho. Não confundir com os requisitos do software. Apenas o que se pretendia atingir originalmente.] %

Este estágio está diretamente relacionado com o produto NkaAcademies, que se trata de uma solução web completa para a gestão de processos de formação.

%1ª opçao
%Deste modo, os principais objetivos estipulados foram a migração completa do produto NkaAcademies para as versões mais recentes do PHP e do MySQLi e posteriormente o desenvolvimento de novas funcionalidades.

%2ª opçao
Pretende-se efetuar a migração completa do produto NkaAcademies para as versões mais recentes do PHP e do MySQLi,no entanto será necessário resolver alguns problemas de compatibilidade que poderão surgir consequentemente a esta alteração e numa fase mais avançada do estágio, incorporar novas funcionalidades ao produto.

\section{Contexto}
 %[No caso de um estágio, é nesta secção que se deverá falar da empresa em que o estágio foi realizado. Se o projeto desenvolvido faz parte de um projeto mais amplo, faz sentido que se documente os objetivos do projeto com um todo, de modo que o leitor consiga perceber onde o trabalho realizado encaixa.] %
\par Este projeto foi realizado no âmbito de estágio curricular da Licenciatura em Engenharia de Sistemas Informáticos, na empresa NKA - \textit{New Knowledge Advice} em Braga.
\par O estágio curricular teve uma duração de aproximadamente quatro meses, tendo iniciado a 22 de fevereiro e terminado a 9 junho de 2021, de segunda a quinta feira num regime maioritariamente de teletrabalho, com exceção de algumas semanas em regime presencial.
\par A NKA - \textit{New Knowledge Advice} é uma empresa tecnológica sediada em Braga, fundada em 2011 que concebe e desenvolve soluções globais vocacionadas para a otimização de cada negócio, por meio de projetos específicos, fornecendo as soluções mais adequadas às crescentes exigências dos mercados. A empresa dedica-se ainda ao desenvolvimento de software, à implementação de sistemas e à consultoria e formação. % incluir ref ao site da nka (??)

\section{Plano de trabalhos}

No âmbito do estágio, foram identificadas as seguintes tarefas:

\begin{itemize}
    \item  Conhecer a empresa, tecnologias e projeto a desenvolver - Migração do produto NkaAcademies para as versões mais recentes do PHP e MySQLi;
    \item  Investigação sobre PHP e MySQLi - Leitura de CHANGELOG das versões anteriores. Análise de funções \textit{deprecated}.
    \item  Alteração do código para a versão 8 PHP.
    \item  Testes e correção de erros de compatibilidade
    \item  Conhecer o projeto a desenvolver - módulo auxiliar ao processo de gestão de formação.
    \item Levantamento de requisitos e funcionalidades
    \item  Desenvolvimento do módulo - Filtro de pesquisa e alteração manual e automática de documentos associados à gestão de formação.
    \item Testes e correção de erros
    \item  Preparação da aplicação - Alteração de ligações para cloud. Limpeza de código.
    \item Atualização das alterações no servidor
\end{itemize}
Para planear a execução das tarefas, foi elaborado um diagrama de Gantt de acordo com a Figura - ~\ref{fig:gantt}.

\begin{center}
        \includegraphics[width=\textwidth,height=\textheight,keepaspectratio]{images/gantt.png}
        \captionof{figure}{Diagrama de Gantt}
        \label{fig:gantt}
\end{center}


\section{Estrutura do documento}
 [A última secção da introdução deve explicar a estrutura do documento: quais são só capítulos existentes (para além do primeiro) e o que será discutido em cada um desses capítulos. A estrutura típica de um relatório de desenvolvimento de software é:

 Introdução, com um breve resumo do que se pretende atingir, e uma descrição clara dos objetivos;

\begin{enumerate}
    \item Análise ao problema, que poderá incluir uma análise ao estado da arte ou ao modelo de negócio onde se pretende intervir;
    \item Análise e modelação do sistema, em que sejam levantados sistematicamente os requisitos, descritos diagramas de caso de uso e de atividade (que descrevam/formalizem o modelo de negócio).
    \item Implementação, em que se descrevam as tecnologias escolhidas (e se justifiquem), e se refira detalhes sobre a implementação.
    \item Análise de resultados e testes, seja uma análise/avaliação aos resultados obtidos, sejam testes de usabilidade ou unitários ao trabalho desenvolvido.
    \item Conclusão.]
\end{enumerate}{}


\chapter{Processo de Reengenharia}


\section{Sistemas AS-IS}
\subsection{Descrição da Arquitetura/tecnologias utilizadas}
\subsubsection{Tecnologias e Plataformas}
\label{tecnologias}
\par Neste ponto serão descritas as tecnologias e plataformas utilizadas ao longo deste trabalho, tais como \acrshort{xampp}, Laragon, Visual Studio Code.
\par Para a realização deste projeto foram utilizadas as plataformas e tecnologias requeridas pela empresa para facilitar a implementação e integração do projeto, e para tal foi necessário proceder à sua aprendizagem.
\par De seguida vão ser descritas de forma sucinta as tecnologias e plataformas utilizadas ao longo do projeto.\newline


\textbf{XAMPP}

O \acrshort{xampp} é formado por um pacote que inclui, base de dados MySQL, servidor web Apache e interpretadores para as linguagens de script. É essencialmente utilizado pelos desenvolvedores que pretendem criar um servidor web local no seu próprio computador, com a finalidade de realizar testes sem necessitar de acesso à rede \citep{xampp}.\newline


\textbf{Visual Studio Code}

O VsCode é um editor de código-fonte simplificado com suporte para operações de desenvolvimento como \textit{debugging}, execução de tarefas e controlo de versões \citep{vscode}.\newline % vs code FAQ - https://code.visualstudio.com/docs/supporting/FAQ

\quad \textbf{Xdebug}

\quad O Xdebug é uma extensão para \acrshort{php} que fornece uma variedade de recursos para melhorar a experiência de desenvolvimento de \acrshort{php} \citep{xdebug}.\newline


\textbf{Laragon}

O Laragon é uma maneira rápida e fácil de criar um ambiente de desenvolvimento isolado no Windows. Inclui Mysql, PHP Memcached, Redis, Apache \citep{laragon}.\newline

% keycdn https://www.keycdn.com/blog/web-development-tools
%\subsection{Identificação das lacunas/melhorias a realizar}
\section{Sistemas TO-BE}
%\subsection{Requisitos Funcionais}
%\subsection{Requisitos técnicos e restrições ao desenvolvimento}



%---------------------
\subsection{Identificação das lacunas/melhorias a realizar}


Atualmente, no mundo em que vivemos, \textit{updates} tecnológicos são lançados praticamente todos os dias relativos a sistemas de informação. É do maior interesse da parte de uma empresa, manter o seu produto atual e competitivo.

De momento, o produto NkaAcademies encontra-se desatualizado, utilizando versões antigas de \acrshort{php}(5.6) e MySQL. Foi então proposto desenvolver a migração de todo o produto NkaAcademies para as versões mais recentes, suportadas no \textit{host} utilizado pela empresa, do \acrshort{php}(5.6 -> 8.0.2) e do MySQL com o objetivo de atualizar o produto e também, em casos futuros o uso de novas funcionalidades que surgiram apenas em versões superiores à que o produto estava desenvolvido.

As melhorias a realizar passaram maioritariamente pela substituição total da biblioteca \acrshort{php} MySQL para MySQLi, uma vez que grande parte das funções que fazem parte da biblioteca MySQL foram descontinuadas / \textit{deprecated}, enquanto noutros casos apenas foram feitas alterações ao nível da chamada da função ou dos parâmetros (e ordem) recebidos. Posto isto, procedeu-se à sua alteração total no produto NkaAcademies, sendo que no caso de funções \textit{deprecated} foram implementadas alternativas \textit{hard-code} em \acrshort{php}.

%----

\subsection{Requisitos}

% - req funcionais e nao funcionais
Uma vez que se trata de uma migração completa de um produto pré-existente e que apenas foram efetuadas alterações ao nível de chamadas de funções, parâmetros recebidos por funções e correções de erros de compatibilidade concluí-se que os requisitos são os mesmos que foram inicialmente estipulados para o desenvolvimento do produto NkaAcademies.


Como tal, o requisito mais importante definido para o cumprimento desta fase tem a ver com a compatibilidade das alterações feitas. As novas funções e todas as modificações realizadas ao nível do produto devem ser implementadas tendo em vista a sua compatibilidade com as versões anteriores, que neste caso são as versões \acrshort{php} 5.6 e 7 e a compatibilidade com outras funcionalidades já existentes no produto garantindo a operabilidade e o bom funcionamento da aplicação na nova versão tecnológica.


%----

\subsection{Alterações no desenvolvimento da solução}

Neste capítulo serão apresentadas algumas das principais alterações implementadas no produto NkaAcademies na fase da migração da aplicação para as versões mais recentes do \acrshort{php} e MySQL.


As principais alterações efetuadas no projecto NkaAcademies foram ao nível da chamadas de funções, implementação de novos métodos devido à descontinuidade de funções da biblioteca MySQL, parâmetros (e ordem) recebidos por funções e correções de erros de compatibilidade entre versões.

\subsubsection{\textit{mysql\_select\_db vs mysqli\_select\_db}}

Neste caso, será apresentado um excerto de código relativo a uma função que foi atualizada na biblioteca MySQLi, recebendo o mesmo tipo de parâmetros mas que trocou a ordem de como eram incorporados na função.

A função \textit{mysql\_select\_db} (listagem~\ref{lst:6}) seleciona uma base de dados do tipo MySQL.

\begin{lstlisting}[language={php},
                   caption={Função mysql\_select\_db.},
                   label=lst:6]
function mysql_select_db($database, $link_identifier=NULL):bool

\end{lstlisting}

A função da Listagem~\ref{lst:6} recebe como argumentos:
\begin{itemize}
  \item \textbf{\$database}: nome da base de dados que deverá ser selecionada.
  \item \textbf{\$link\_identifier}: conexão ao MySQL.
\end{itemize}


A função \textit{mysqli\_select\_db} (listagem~\ref{lst:7}) seleciona a base de dados a ser utilizada ao realizar \textit{queries} do tipo MySQLi.

\begin{lstlisting}[language={php},
                   caption={Função mysqli\_select\_db.},
                   label=lst:7]
    	function mysqli_select_db($mysql, $database):bool

\end{lstlisting}

A função da Listagem~\ref{lst:7} recebe como parâmetros:
\begin{itemize}
  \item \textbf{\$mysql}: \textit{connection string} de acesso à base de dados.
  \item \textbf{\$database}: nome da base de dados que deverá ser selecionada.
\end{itemize}

%-----

\subsubsection{\textit{mysql\_result vs nka\_mysqli\_result}}

Neste exemplo será apresentado nos excertos de código a alteração que teve de ser feita quando a função da biblioteca MySQL não foi continuada na biblioteca MySQLi, fazendo com que tivesse de ser criada uma equivalente.

A função \textit{mysql\_result} (listagem~\ref{lst:4}) obtém os dados de um determinado resultado.

\begin{lstlisting}[language={php},
                   caption={Função mysql\_result.},
                   label=lst:4]
    	function mysql_result($result, $row, $field=0):string

\end{lstlisting}

A função da Listagem~\ref{lst:4} recebe como argumentos:
\begin{itemize}
  \item \textbf{\$result}: resultado da \textit{query} \acrshort{sql} a ser executado.
  \item \textbf{\$row}: número da linha do resultado.
  \item \textbf{\$field}: nome do campo a ser procurado (tabela).
\end{itemize}


A função \textit{nka\_mysqli\_result} (listagem~\ref{lst:1}), utilizada para substituir a função \textit{deprecated mysql\_result} foi implementada recebendo os mesmos argumentos que a anterior, e retornando uma \textit{string}.

\begin{lstlisting}[language={php},
                   caption={Função para substituir mysql\_result.},
                   label=lst:1]
      if (!function_exists('nka_mysqli_result')) {
      	function nka_mysqli_result($res, $row, $field=0) {
      	  $res->data_seek($row);
      	  $datarow = $res->fetch_array();
      	  return $datarow[$field];
      	}
      }
\end{lstlisting}

A função da Listagem~\ref{lst:1} recebe como argumentos:
\begin{itemize}
  \item \textbf{\$result}: resultado da \textit{query} \acrshort{sql} a ser executado.
  \item \textbf{\$row}: número da linha do resultado.
  \item \textbf{\$field}: nome do campo a ser procurado (tabela).
\end{itemize}


\subsubsection{\textit{mysql\_field\_name vs mysqli\_field\_name}}

Nas Listagens~\ref{lst:5} e \ref{lst:2} está mais uma vez demonstrado o caso em que uma função da biblioteca MySQL é descontinuada para MySQLi e teve de se proceder à sua implementação.

A função \textit{mysql\_field\_name} (listagem~\ref{lst:5}) obtém o nome do campo especificado num resultado.


\begin{lstlisting}[language={php},
                   caption={Função mysql\_field\_name.},
                   label=lst:5]
  function mysql_field_name($result, $field_offset):string|false

\end{lstlisting}

A função da Listagem~\ref{lst:5} recebe como parâmetros:
\begin{itemize}
  \item \textbf{\$result}: resultado da \textit{query} \acrshort{sql} a ser executada.
  \item \textbf{\$field\_offset}: index/\textit{offset} de um campo numérico.
\end{itemize}


A função \textit{mysqli\_field\_name} (listagem~\ref{lst:2}), utilizada para substituir a função \textit{deprecated mysql\_field\_name} foi implementada no projeto recebendo os mesmos parâmetros que a anterior, e retornando uma \textit{string} ou \textit{false}.

\begin{lstlisting}[language={php},
                   caption={Função para substituir mysql\_field\_name.},
                   label=lst:2]
  if (!function_exists('mysqli_field_name')) {
    function mysqli_field_name($result, $field_offset){
$properties = mysqli_fetch_field_direct($result, $field_offset);
  return is_object($properties) ? $properties->name : false;
  }
  }
\end{lstlisting}

A função da Listagem~\ref{lst:2} recebe como parâmetros:
\begin{itemize}
  \item \textbf{\$result}: resultado da \textit{query} \acrshort{sql} a ser executada.
  \item \textbf{\$field\_offset}: index/\textit{offset} de um campo numérico.
\end{itemize}

%--

\subsubsection{\textit{Improvement} da função \textit{nka\_mysqli\_result}}

Neste caso está demonstrado um exemplo \textit{improvement} de uma função que já tinha sido desenvolvida para substituir uma função descontinuada da biblioteca MySQL.

A função \textit{nka\_mysqli\_result\_v2} (listagem~\ref{lst:3}), foi um \textit{improvement} da função previamente implementada \textit{nka\_mysqli\_result}. Esta alteração deveu-se ao facto de alguns erros na incompatibilidade das diferentes versões(7 e 8) do \acrshort{php} em que o \textit{\$datarow} chegava \textit{empty}, enquanto que na versão \acrshort{php}5.6 a mesma função não apresentava erro. Verificou-se que a mesma função se comportava de forma diferente de versão para versão e como tal, procedeu-se à sua remodelação passando agora a retornar \textit{string} ou \textit{null} resolvendo o problema.


\begin{lstlisting}[language={php},
                   caption={Improvement da função nka\_mysql\_result.},
                   label=lst:3]
      if (!function_exists('nka_mysqli_result_v2')) {
        function nka_mysqli_result_v2($res, $row, $field=0) {
      	  $res->data_seek($row);
      	  $datarow = $res->fetch_array();
      	  $retval = null;
      	  $cond = ! empty($datarow);
      	  if($cond){
      		    $retval = $datarow[$field];
      	  }
      	  return $retval;
      	}
      }
\end{lstlisting}

A função da Listagem~\ref{lst:3} recebe como argumentos:
\begin{itemize}
  \item \textbf{\$result}: resultado da \textit{query} \acrshort{sql} a ser executado.
  \item \textbf{\$row}: número da linha do resultado.
  \item \textbf{\$field}: nome do campo a ser procurado (tabela).
\end{itemize}


% - falar que no 5.6 nka_mysqli_result nao acusava erro mas em versoes superiores 7 e 8 dava erro. A mesma funçao comportava-se diferente nas 2 versões desenvolvidas.


\chapter{Desenvolvimento da aplicação}
\label{desenv}
Este capítulo apresenta o trabalho desenvolvido numa segunda parte do estágio, o desenvolvimento de um módulo a implementar no produto NkaAcademies. Foi necessário analisar requisitos, tratar da modelação do produto e da sua implementação.

\section{Análise e Especificações}

Na identificação de requisitos são enunciadas todas as funcionalidades, de forma enumerada, que se pretendem implementar no \textit{software}.
Os diferentes requisitos, funcionais e não funcionais, encontram-se classificados pela metodologia \textit{\gls{moscoww}} que identifica os requisitos de elevada prioridade até aos menos prioritários.

Cada tipo de requisito classificado (\textit{\textbf{M}ust}, \textit{\textbf{S}hould}, \textit{\textbf{C}ould}, e \textit{\textbf{W}ould}) representam diferentes significados face ao foco e importância que deverá ser levada em conta durante o processo de desenvolvimento.

Assim sendo:


\begin{enumerate}
  \item \textbf{\textit{Must}}: classificados como requisitos mais críticos ou indispensáveis para o produto, pretende-se o total foco sobre este tipo de requisitos durante o processo de desenvolvimento;
  \item \textbf{\textit{Should}}: são tão importantes como os requisitos classificados como \textit{Must Have}, contudo não são considerados tão priorizados;
  \item \textbf{\textit{Could}}: entende-se como requisitos desejáveis, mas também não são necessários;
  \item \textbf{\textit{Would}}: estes são os requisitos menos críticos e com menor valor, podendo ser implementados ou não.
\end{enumerate}

\subsection{Requisitos Funcionais}

Os requisitos funcionais dizem respeito a todas as funcionalidades ou pressupostos de como o sistema deve reagir a entradas específicas e como se deverá comportar. Estes tipos de requisitos podem-se entender como funcionalidades que o utilizador pode interagir de forma direta e mais visível.

A Tabela~\ref{tab:1} - enuncia os requisitos funcionais estipulados para o sistema.

\begin{longtable}{|l|l|l|l|}

\hline

\textbf{\#} & \textbf{Requisito} & \textbf{Descrição} & \textbf{Prioridade} \\ \hline

RF01 & Pesquisar formação  & \vtop{\hbox{\strut O sistema tem que permitir ao utilizador} \hbox{\strut a procura pelo código da formação ou código} \hbox{\strut de curso}} & MUST \\ \hline
RF02 & Escolher declaração & \vtop{\hbox{\strut O sistema tem que permitir ao utilizador} \hbox{\strut a escolha da declaração que pretende } \hbox{\strut imprimir / preencher.}} & MUST \\ \hline
RF03 & Editar manualmente  & \vtop{\hbox{\strut O sistema deverá permitir ao utilizador a }\hbox{\strut edição manual dos campos a preencher}} & MUST \\ \hline
RF04 & Criar campos  & \vtop{\hbox{\strut O sistema deve permitir a criação de }\hbox{\strut novos campos a serem adicionados às }\hbox{\strut declarações (\textit{template})}} & COULD \\ \hline
RF05 & Editar automaticamente  & \vtop{\hbox{\strut O sistema deverá permitir ao utilizador}\hbox{\strut a edição automática dos campos a preencher,}\hbox{\strut conforme os resultados da procura}}  & MUST \\ \hline
RF06 & Imprimir documento & \vtop{\hbox{\strut O sistema deverá permitir ao utilizador}\hbox{\strut a impressão dos documentos editados.}} & SHOULD \\ \hline


\caption{Requisitos funcionais}\\
\label{tab:1}\\
\end{longtable}

% ------ Req nao Funcionais

\subsection{Requisitos Não Funcionais}

Requisitos não funcionais são as propriedades comportamentais relacionadas ao uso da aplicação em termos de desempenho, usabilidade, confiabilidade, segurança, disponibilidade, manutenção e tecnologias envolvidas. Estes requisitos dizem respeito a como as funcionalidades serão entregues ao utilizador do software.

A Tabela~\ref{tab:2} - encontram-se especificados os requisitos não funcionais do sistema.

\begin{longtable}{|l|l|l|l|}

\hline

\textbf{\#} & \textbf{Descrição} & \textbf{Categoria} & \textbf{Prioridade} \\ \hline

RNF01 & \vtop{\hbox{\strut As novas funções deverão ser implementadas de}\hbox{\strut forma a que sejam compatíveis com outras}\hbox{\strut funcionalidades já existentes}} & Compatibilidade & MUST \\ \hline
RNF02 & \vtop{\hbox{\strut O módulo deverá ser implementado em JavaScript,}\hbox{\strut PHP com acesso à base de dados MySQL}} &  Operacional  & MUST \\ \hline
RNF03 & \vtop{\hbox{\strut Apenas utilizadores autenticados e com permissão}\hbox{\strut devem ter acesso módulo desenvolvido}} & Segurança & MUST \\ \hline

\caption{Requisitos não funcionais}\\
\label{tab:2}\\
\end{longtable}


\section{Arquitetura técnica}

As tecnologias e linguagens utilizadas para o desenvolvimento deste módulo a adicionar ao produto NkaAcademies foram essencialmente descritas no Capítulo~\ref{tecnologias}, com adição de JavaScript e \gls{bootstrap} para desenvolver a componente web da aplicação.

\subsection{Diagrama de Atividades}

Para descrever o processo de negócio foi utilizado o diagrama de atividade, uma vez que permite modelar o comportamento do sistema incluindo a sequência e as condições de execução de ações.

A Figura~\ref{fig:act} demonstra o diagrama de atividades desenvolvido para o sistema.

\begin{center}
        \includegraphics[width=\textwidth,height=\textheight,keepaspectratio]{images/ActivityDiagram1.jpg}
        \captionof{figure}{Diagrama de Atividades}
        \label{fig:act}
\end{center}

Recorrendo à visualização do diagrama da Figura~\ref{fig:act}, é possível perceber os diferentes passos que correspondem à atividade geral do desenvolvimento desta fase do projeto, o preenchimento de documentos de apoio aos processos de gestão de formação. As atividades foram então: filtrar a procura a um resultado, escolher um \textit{template} / declaração predefinido, proceder ao seu preenchimento manual/automático e posteriormente à sua impressão caso seja pretendido.

No desenho deste diagrama está subentendido que o cliente está devidamente autenticado e verificadas as suas permissões no acesso a este tipo de ferramenta dentro da aplicação.

\subsection{Estrutura do código}
% colocar imagem a apresentar a estrutura do código.
Neste sub-capítulo será apresentada a estrutura do código desenvolvido, bem como alguns excertos de código para exemplificar as diferentes camadas existentes no projeto: base de dados, \textit{back-end} e \textit{front-end}.

A Figura~\ref{fig:estrutura} apresenta um esquema gráfico, de forma a que se perceba melhor esta estrutura e as relações entre camadas.

\begin{center}
        \includegraphics[width=\textwidth,height=\textheight,keepaspectratio]{images/estrutura.png}
        \captionof{figure}{Estrutura do código}
        \label{fig:estrutura}
\end{center}

\subsubsection{Base de Dados}

%A base de dados utilizada para o desenvolvimento deste projeto
Quanto à base de dados utilizada para o desenvolvimento do projeto, não foi feito qualquer tipo de modelação uma vez que foi reaproveitada a base de dados já existente do produto NkaAcademies tendo sido apenas necessário acrescentar alguns dados em tabelas(\textit{templates}) já existentes a fim de testes e futura implementação.

% nao foi feita a modelaçao ao nivel da bd .. apenas foram acrescentadas linhas em algumas colunas (templates) para testes e futura implementação.

\subsubsection{\textit{back-end}}

A parte do \textit{back-end} da aplicação foi desenvolvida em \acrshort{php}. O principal objetivo do trabalho realizado no \textit{back-end} diz respeito a interações com a base de dados, ou seja, tudo que envolva \textit{queries\acrshort{sql}} e ligações a bases de dados são definidas neste sector.

Para uma melhor compreensão, será apresentada na Listagem~\ref{lst:99} um exemplo de uma interação com a base de dados onde basicamente é retornado o conteúdo do \textit{template} selecionado pelo utilizador (\textit{\$\_POST["diploma"]}). Caso se verifique que o \textit{iduser} / cliente existe e o template selecionado está definido na base de dados, é devolvido o seu conteúdo para a interface web.


\begin{lstlisting}[language={php},
                   caption={Exemplo de request à base de dados},
                   label=lst:99]
case 2073:

$result = mysqli_query($con,"SELECT * from xxx.templates
where diploma='".$_POST["diploma"]."'and iduser='$varemp'");

if (mysqli_num_rows($result)!=0)
{
echo str_replace("table table-bordered",
"",utf8_encode(nka_mysqli_result($result,0,"body")));
}
else
{
	echo "Sem Template Definido";
}

mysqli_close($con);

break;
\end{lstlisting}
%-- apresentar como são feitas as interações com a base de dados
\subsubsection{\textit{front-end}}

A parte do \textit{front-end} foi desenvolvida em \acrshort{html}, recorrendo a \textit{modals} do \gls{bootstrap} para implementar a interface web e JavaScript que contém as funções que são o elo de ligação entre a interface web e o \textit{back-end}, que permite atualizar e tornar dinâmica a parte gráfica/visual do sistema.

A Listagem~\ref{lst:98} demonstra a implementação de uma função JavaScript que complementa a ação realizada ao \textit{back-end} na listagem~\ref{lst:99}.

De salientar ainda, a importância do uso de \textit{jQuery.ajax()} que permite utilizar JavaScript para enviar \textit{asynchronous http request} e obter resposta em vários formatos, e também para atualizar partes de uma página web (que use JavaScript) sem ter que recarregar a página web completa \citep{jquery}.

Assim sendo, o \textit{jquery} neste caso é responsável por passar o parâmetro \textbf{diploma} para o \textit{back-end} de forma a ser processada a query\acrshort{sql} correspondente à declaração que o cliente selecionou.



\begin{lstlisting}[language={php},
                   caption={Função JavaScript utilizando jQuery.ajax()},
                   label=lst:98]
function carregatexto(){
    var resultado = $['ajax']({
    type: 'POST',
    url: 'phpScripts/dbload.php',
    data: 'kind=2073&diploma='
        +document.getElementById("template").value,
    async: false })['responseText'];

	$(".summernote").summernote("code", resultado);

	initcampos();
}
\end{lstlisting}


Na Figura~\ref{fig:web} está representada a interface web resultante das listagens de código descritas acima nos pontos \ref{lst:99} e \ref{lst:98}.

De salientar o uso do \textit{summernote} que é uma biblioteca do JavaScript que ajuda na criação de um editor \acrshort{wysiwyg} online \citep{summernote}.


\begin{center}
        \includegraphics[width=\textwidth,height=\textheight,keepaspectratio]{images/frontend.png}
        \captionof{figure}{Interface Web - template teste}
        \label{fig:web}
\end{center}

% -- apresentar listagem de codigo de como é feito o request ao backend jquery.ajax()
% -- funçoes JS que depois sao chamadas e sao responsaveis por atualizar e tornar responsive a parte da interface web


\chapter{Análise dos Resultados}
\label{analise}

Neste capítulo encontra-se uma revisão geral ao que foi abordado, através de uma apresentação de resultados obtidos, uma avaliação do nível de maturidade e cumprimentos dos objetivos bem como de algumas dificuldades ultrapassadas no desenvolvimento do projeto.

\section{Apresentação da solução desenvolvida}
\label{app}

Nesta secção serão apresentadas as soluções desenvolvidas divididas em dois sub-capítulos referentes às duas partes que compuseram o estágio, o processo de reengenharia e o desenvolvimento do novo módulo.

\subsection{Solução desenvolvida - Processo de reengenharia}

A primeira parte do projeto de estágio prende-se no processo de reengenharia, descrito no capítulo~\ref{migracao}, que teve como principal objetivo a migração completa do produto NkaAcademies para as versões mais recentes do \acrshort{php} e MySQL.

Este tipo de alteração torna-se um pouco complicado de demonstrar, mas a migração total da solução NkaAcademies foi realizada e está neste momento, aos poucos, a ser introduzida no servidor \textit{host} responsável pelo armazenamento do produto para não causar conflitos, uma vez que se pretende garantir o contínuo bom funcionamento da solução \textit{online}.

Segue na Figura~\ref{fig:exemplo} uma demonstração do painel "Gestão de Formação"  funcional de acordo com as novas atualizações tecnológicas.

\begin{center}
        \includegraphics[width=\textwidth,height=\textheight,keepaspectratio]{images/Exemplo.png}
        \captionof{figure}{Menu Gestão Formação}
        \label{fig:exemplo}
\end{center}

\subsection{Solução desenvolvida - Desenvolvimento do novo módulo}

A segunda parte do projeto de estágio estava direcionada para a implementação de um módulo / funcionalidade a adicionar ao produto NkaAcademies, um sistema de apoio à gestão de processos de formação - preenchimento de dossiers pedagógicos, de forma a passar a customização e a parte de gerar estes dossiers para o lado do cliente.

\begin{center}
        \includegraphics[width=\textwidth,height=\textheight,keepaspectratio]{images/dadoscurso.png}
        \captionof{figure}{Menu dados curso}
        \label{fig:dadoscurso}
\end{center}

Deste modo, o cliente começaria com algo como a Figura~\ref{fig:dadoscurso}, onde ao clicar no botão \textbf{Procurar} seria lançado um \textit{pop-up}, que na verdade é apenas a utilização de \textit{modals} do \gls{bootstrap} como é possível observar na Figura~\ref{fig:procurar}.

\begin{center}
        \includegraphics[width=\textwidth,height=\textheight,keepaspectratio]{images/procurar.png}
        \captionof{figure}{Menu procurar}
        \label{fig:procurar}
\end{center}

Após ser aberto o "Menu Procurar", o cliente deverá pesquisar pelo código da formação ou pelo nome do curso para que a \textit{drop-down box} "Escolha uma opção" seja preenchida com os resultados obtidos da pesquisa realizada. Conforme altera a opção obtida na \textit{drop-down box}, todos os campos são dinâmicos e deverão se ajustar à informação que está a ser selecionada, como é possível visualizar na Figura~\ref{fig:procurarpreenchido}.

\begin{center}
        \includegraphics[width=\textwidth,height=\textheight,keepaspectratio]{images/procurardados.png}
        \captionof{figure}{Menu procurar preenchido}
        \label{fig:procurarpreenchido}
\end{center}

Uma vez selecionados os campos que pretende visualizar, e selecionando o botão "Escolher", os dados (Figura~\ref{fig:dadoscursopreenchido}) são automaticamente carregados para fora do \textit{modal}, sendo preenchidos nos locais anteriormente vazios na Figura~\ref{fig:dadoscurso}.

\begin{center}
        \includegraphics[width=\textwidth,height=\textheight,keepaspectratio]{images/dadoscursopreenchido.png}
        \captionof{figure}{Menu dados curso preenchido}
        \label{fig:dadoscursopreenchido}
\end{center}

Depois de filtrados os resultados apenas a um, resta ao cliente selecionar o tipo de \textit{template} que deseja observar, que neste caso apenas está demonstrado um para efeitos de teste (Figura~\ref{fig:web}). Se optar por "Preencher Automaticamente", como ilustrado na Figura~\ref{fig:preencheauto} os campos serão automaticamente substituídos pelos dados recolhidos acima, na Figura~\ref{fig:dadoscursopreenchido}.

\begin{center}
        \includegraphics[width=\textwidth,height=\textheight,keepaspectratio]{images/preencheauto.png}
        \captionof{figure}{Preenchimento automático}
        \label{fig:preencheauto}
\end{center}

Caso o cliente opte por fazer o preenchimento das declarações manualmente, também é possível, como observado na Figura~\ref{fig:preenchemanual} apenas precisando de alterar para a informação que pretende e "Preencher".

\begin{center}
        \includegraphics[width=\textwidth,height=\textheight,keepaspectratio]{images/preenchemanual.png}
        \captionof{figure}{Menu preenchimento manual}
        \label{fig:preenchemanual}
\end{center}

Por fim, as alterações indicadas na Figura~\ref{fig:preenchemanual} são feitas e o resultado da declaração é atualizado, como é possível observar na Figura~\ref{fig:resultpreencheman}.

\begin{center}
        \includegraphics[width=\textwidth,height=\textheight,keepaspectratio]{images/resultpreencheman.png}
        \captionof{figure}{Preenchimento manual}
        \label{fig:resultpreencheman}
\end{center}

\section{Avaliação do nível de maturidade e cumprimento dos objetivos}

Após a apresentação da solução desenvolvida (sub-capítulo~\ref{app}) foi elaborada uma análise de comparação entre o que estava inicialmente estipulado para o estágio e o que foi desenvolvido. No geral, os objetivos e requisitos definidos foram cumpridos na sua totalidade.

Durante o todo o processo de estágio houve necessidade de fazer investigação sobre as principais alterações de cada versão da linguagem \acrshort{php}, bem como da forma como as tecnologias operam, uma vez que não detinha os principais conhecimentos sobre as linguagens utilizadas. Como tal, o desenvolvimento deste projeto de estágio exigiu uma elevada auto motivação para o cumprimento dos objetivos propostos.

%---

\section{Principais dificuldades encontradas}

Durante a migração da aplicação entre as versões do \acrshort{php} 5.6 -> 8, mesmo apesar da investigação realizada a fim de perceber as principais mudanças entre versões, as funções e bibliotecas \textit{deprecated} / descontinuadas houveram bastantes problemas em conseguir uma implementação de forma a garantir a compatibilidade e operabilidade do produto nas versões mais recentes.

Trata-se de uma linguagem de programação que sofreu algumas alterações desde a versão em que estava implementada no produto, e acima de tudo de uma versão recente da tecnologia que nem um ano possuí, ou seja, mesmo a nível de documentação e \textit{CHANGELOG} muito pouco evoluído.

Outra das adversidades que pode ser considerada um entrave ao projeto inicialmente foi o facto de estar a trabalhar com tecnologias novas em termos de conhecimento próprio, como o \acrshort{php} e JavaScript tendo sido necessário proceder à sua aprendizagem.

%\section{Parecer do supervisor}

\chapter{Conclusão Final}
\label{conc}
\section{Principais conclusões do estágio}

Neste período de quatro meses de estágio, o contacto com novas tecnologias, a integração numa empresa e a experiência de desenvolver projectos inseridos em ambientes de grande escala são algumas das competências adquiridas que completam a formação académica.

O estágio ofereceu-me uma experiência inicial muito importante no desenvolvimento de competências úteis para o meu futuro profissional, uma vez que desde muito cedo fui parte integrante em algumas reuniões de equipa de projetos, ou seja, experienciava de perto a maneira de como um produto era trabalhado e organizado até ao seu desenvolvimento e foram-me transmitidos muitos valores profissionais dentro da empresa.

Por fim, elaborando uma análise global ao trabalho desenvolvido é visível todo o trabalho realizado ao longo destes quatro meses. O objetivo, na sua totalidade, ainda se encontra um pouco longe do atingido, contudo o objetivo delineado para o estágio curricular foi cumprido uma vez que os requisitos impostos no ínicio do estágio foram concluídos, embora a última parte do desenvolvimento do módulo de apoio à gestão dos processos de formação - dossiers pedagógicos continue em fase de protótipo e como futura implementação no produto NkaAcademies.


%------ +1 paragrafo??!?!?!
\section{Pontos a melhorar/completar}

Neste projeto, os pontos a melhorar estão mais direcionados ao desenvolvimento da ferramenta de apoio ao preenchimento de documentos associados aos processos de formação, ou seja, à segunda parte do estágio.

O módulo desenvolvido ainda está em fase protótipo e como potencial implementação futura no produto NkaAcademies, e sendo que existiram alguns requisitos estipulados, de menor prioridade de acordo com a metodologia \acrshort{moscow} que não foram implementados devido à falta de tempo considera-se que estes são alguns dos pontos que podem ser completados futuramente.



\bibliography{biblio}

\end{document}
