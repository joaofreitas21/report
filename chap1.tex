
\chapter{Introdução}


Acompanhar as novas tecnologias é fundamental para um negócio. É preciso investir em inovação para manter um negócio no mercado competitivo. A tecnologia, aliada a uma boa atualização, consegue proporcionar soluções eficientes às empresas.

Atualmente, no mundo em que vivemos, \textit{updates} tecnológicos são lançados praticamente todos os dias relativos a sistemas de informação. É do maior interesse da parte de uma empresa, manter o seu produto atual e competitivo.

De momento, o produto NkaAcademies por parte da NKA - \textit{New Knowledge Advice} encontra-se desatualizado a nível tecnológico, por se tratar de uma solução com algum tempo de desenvolvimento e permanência no mercado tecnológico como um forte produto a nível de gestão de processos de formação.

Como tal, este estágio visa principalmente a alteração das tecnologias utilizadas no produto NkaAcademies de forma a que a solução esteja preparada para futuras implementações desenvolvidas apenas em versões mais recentes e consiga assim alcançar todo o seu potencial para continuar a conquistar o mercado como um produto de excelência na gestão dos processos de formação.

%produto alcançar o seu potencial máximo e inovar para continuar a conquiustar o mercado na gestao de processos de formaçao


\section{Objetivos}

% - [Numa pequena secção da introdução liste, cuidadosamente, os objetivos do trabalho. Não confundir com os requisitos do software. Apenas o que se pretendia atingir originalmente.] %

Este estágio está diretamente relacionado com o produto NkaAcademies, que se trata de uma solução web completa para a gestão de processos de formação.

%1ª opçao
%Deste modo, os principais objetivos estipulados foram a migração completa do produto NkaAcademies para as versões mais recentes do PHP e do MySQLi e posteriormente o desenvolvimento de novas funcionalidades.
% no entanto será necessário resolver alguns problemas de compatibilidade que poderão surgir consequentemente a esta alteração
%2ª opçao
Pretende-se efetuar a migração completa do produto NkaAcademies para as versões mais recentes do PHP e do MySQLi e numa fase mais avançada do estágio, incorporar novas funcionalidades ao produto.

%corrigir !?!

\section{Contexto}
 %[No caso de um estágio, é nesta secção que se deverá falar da empresa em que o estágio foi realizado. Se o projeto desenvolvido faz parte de um projeto mais amplo, faz sentido que se documente os objetivos do projeto com um todo, de modo que o leitor consiga perceber onde o trabalho realizado encaixa.] %
\par Este projeto foi realizado no âmbito de estágio curricular da Licenciatura em Engenharia de Sistemas Informáticos, na empresa NKA - \textit{New Knowledge Advice} em Braga.
\par O estágio curricular teve uma duração de aproximadamente quatro meses, tendo iniciado a 22 de fevereiro e terminado a 9 junho de 2021, de segunda a quinta feira num regime maioritariamente de teletrabalho, com exceção de algumas semanas em regime presencial.
\par A NKA - \textit{New Knowledge Advice} é uma empresa tecnológica sediada em Braga, fundada em 2011 que concebe e desenvolve soluções globais vocacionadas para a otimização de cada negócio, por meio de projetos específicos, fornecendo as soluções mais adequadas às crescentes exigências dos mercados. A empresa dedica-se ainda ao desenvolvimento de software, à implementação de sistemas e à consultoria e formação\citep{nka}.

\section{Plano de trabalhos}

No âmbito do estágio, foram identificadas as seguintes tarefas:

\begin{itemize}
    \item  Conhecer a empresa, tecnologias e projeto a desenvolver - Migração do produto NkaAcademies para as versões mais recentes do PHP e MySQLi;
    \item  Investigação sobre PHP e MySQLi - Leitura de CHANGELOG das versões anteriores. Análise de funções \textit{deprecated}.
    \item  Alteração do código para a versão 8 PHP.
    \item  Testes e correção de erros de compatibilidade
    \item  Conhecer o projeto a desenvolver - módulo auxiliar ao processo de gestão de formação.
    \item Levantamento de requisitos e funcionalidades
    \item  Desenvolvimento do módulo - Filtro de pesquisa e alteração manual e automática de documentos associados à gestão de formação.
    \item Testes e correção de erros
    \item  Preparação da aplicação - Alteração de ligações para cloud. Limpeza de código.
    \item Atualização das alterações no servidor
\end{itemize}
Para planear a execução das tarefas, foi elaborado um diagrama de Gantt de acordo com a Figura - ~\ref{fig:gantt}.

\begin{center}
        \includegraphics[width=\textwidth,height=\textheight,keepaspectratio]{images/unknown.png}
        \captionof{figure}{Diagrama de Gantt}
        \label{fig:gantt}
\end{center}


\section{Estrutura do documento}

Este documento está dividido em vários capítulos e, cada um, dividido em subcapítulos, que explicam sucintamente os diferentes métodos de trabalho e tecnologias utilizadas.

Neste preciso capítulo, foram dadas a conhecer informações relativas ao estágio curricular e à empresa onde foi realizado, os objetivos propostos e a planificação das atividades com recurso ao diagrama de Gantt.

No segundo capítulo, Processo de reengenharia está descrito todo o processo de migração da aplicação NkaAcademies. Aqui serão descritas as tecnologias utilizadas ao longo do desenvolvimento do projeto, identificadas as principais falhas e melhorias a realizar e demonstradas algumas das principais diferenças entre as versões \acrshort{php} 5.6 e 8.

No capítulo \ref{desenv}, Desenvolvimento do novo módulo foram dadas a conhecer todas as análises e especificações do sistema, ou seja, os requisitos funcionais e não funcionais categorizados pela metodologia \acrshort{moscow}. De seguida, foi apresentada a arquitetura do sistema e a estruturação do código de forma a facilitar a compreensão do sistema a desenvolver.

No capítulo \ref{analise} é feita uma análise a tudo o que foi desenvolvido durante o estágio curricular e feita uma observação em relação ao cumprimento dos objetivos inicialmente estipulados.

Por fim, segue-se o capítulo da Conclusão (\ref{conc}) onde é feita uma revisão geral ao estágio desenvolvido, bem como algumas reflexões.
