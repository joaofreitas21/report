
\chapter{Processo de Reengenharia}
\label{migracao}

%\section{Sistemas AS-IS}
\section{Descrição da Arquitetura/tecnologias utilizadas}
\subsection{Tecnologias e Plataformas}
\label{tecnologias}
\par Neste ponto serão descritas as tecnologias e plataformas utilizadas ao longo deste trabalho, tais como \acrshort{xampp}, Laragon, Visual Studio Code.
\par Para a realização deste projeto foram utilizadas as plataformas e tecnologias requeridas pela empresa para facilitar a implementação e integração do projeto, e para tal foi necessário proceder à sua aprendizagem.
\par De seguida vão ser descritas de forma sucinta as tecnologias e plataformas utilizadas ao longo do projeto.\newline


\textbf{XAMPP}

O \acrshort{xampp} é formado por um pacote que inclui, base de dados MySQL, servidor web Apache e interpretadores para as linguagens de script. É essencialmente utilizado pelos desenvolvedores que pretendem criar um servidor web local no seu próprio computador, com a finalidade de realizar testes sem necessitar de acesso à rede \citep{xampp}.\newline


\textbf{Visual Studio Code}

O VsCode é um editor de código-fonte simplificado com suporte para operações de desenvolvimento como \textit{debugging}, execução de tarefas e controlo de versões \citep{vscode}.\newline % vs code FAQ - https://code.visualstudio.com/docs/supporting/FAQ

\quad \textbf{Xdebug}

\quad O Xdebug é uma extensão para \acrshort{php} que fornece uma variedade de recursos para melhorar a experiência de desenvolvimento de \acrshort{php} \citep{xdebug}.\newline


\textbf{Laragon}

O Laragon é uma maneira rápida e fácil de criar um ambiente de desenvolvimento isolado no Windows. Inclui Mysql, PHP Memcached, Redis, Apache \citep{laragon}.\newline

% keycdn https://www.keycdn.com/blog/web-development-tools
%\subsection{Identificação das lacunas/melhorias a realizar}
%\section{Sistemas TO-BE}
%\subsection{Requisitos Funcionais}
%\subsection{Requisitos técnicos e restrições ao desenvolvimento}



%---------------------
\section{Identificação das lacunas/melhorias a realizar}


Atualmente, no mundo em que vivemos, \textit{updates} tecnológicos são lançados praticamente todos os dias relativos a sistemas de informação. É do maior interesse da parte de uma empresa, manter o seu produto atual e competitivo.

De momento, o produto NkaAcademies encontra-se desatualizado, utilizando versões antigas de \acrshort{php}(5.6) e MySQL. Foi então proposto desenvolver a migração de todo o produto NkaAcademies para as versões mais recentes, suportadas no \textit{host} utilizado pela empresa, do \acrshort{php}(5.6 -> 8.0.2) e do MySQL com o objetivo de atualizar o produto e também, em casos futuros o uso de novas funcionalidades que surgiram apenas em versões superiores à que o produto estava desenvolvido.

As melhorias a realizar passaram maioritariamente pela substituição total da biblioteca \acrshort{php} MySQL para MySQLi, uma vez que grande parte das funções que fazem parte da biblioteca MySQL foram descontinuadas / \textit{deprecated}, enquanto noutros casos apenas foram feitas alterações ao nível da chamada da função ou dos parâmetros (e ordem) recebidos. Posto isto, procedeu-se à sua alteração total no produto NkaAcademies, sendo que no caso de funções \textit{deprecated} foram implementadas alternativas \textit{hard-code} em \acrshort{php}.

%----

\section{Requisitos}

% - req funcionais e nao funcionais
Uma vez que se trata de uma migração completa de um produto pré-existente e que apenas foram efetuadas alterações ao nível de chamadas de funções, parâmetros recebidos por funções e correções de erros de compatibilidade concluí-se que os requisitos são os mesmos que foram inicialmente estipulados para o desenvolvimento do produto NkaAcademies.


Como tal, o requisito mais importante definido para o cumprimento desta fase tem a ver com a compatibilidade das alterações feitas. As novas funções e todas as modificações realizadas ao nível do produto devem ser implementadas tendo em vista a sua compatibilidade com as versões anteriores, que neste caso são as versões \acrshort{php} 5.6 e 7 e a compatibilidade com outras funcionalidades já existentes no produto garantindo a operabilidade e o bom funcionamento da aplicação na nova versão tecnológica.


%----

\section{Alterações no desenvolvimento da solução}

Neste capítulo serão apresentadas algumas das principais alterações implementadas no produto NkaAcademies na fase da migração da aplicação para as versões mais recentes do \acrshort{php} e MySQL.


As principais alterações efetuadas no projecto NkaAcademies foram ao nível da chamadas de funções, implementação de novos métodos devido à descontinuidade de funções da biblioteca MySQL, parâmetros (e ordem) recebidos por funções e correções de erros de compatibilidade entre versões.

\subsubsection{\textit{mysql\_select\_db vs mysqli\_select\_db}}

Neste caso, será apresentado um excerto de código relativo a uma função que foi atualizada na biblioteca MySQLi, recebendo o mesmo tipo de parâmetros mas que trocou a ordem de como eram incorporados na função.

A função \textit{mysql\_select\_db} (listagem~\ref{lst:6}) seleciona uma base de dados do tipo MySQL.

\begin{lstlisting}[language={php},
                   caption={Função mysql\_select\_db.},
                   label=lst:6]
function mysql_select_db($database, $link_identifier=NULL):bool

\end{lstlisting}

A função da Listagem~\ref{lst:6} recebe como argumentos:
\begin{itemize}
  \item \textbf{\$database}: nome da base de dados que deverá ser selecionada.
  \item \textbf{\$link\_identifier}: conexão ao MySQL.
\end{itemize}


A função \textit{mysqli\_select\_db} (listagem~\ref{lst:7}) seleciona a base de dados a ser utilizada ao realizar \textit{queries} do tipo MySQLi.

\begin{lstlisting}[language={php},
                   caption={Função mysqli\_select\_db.},
                   label=lst:7]
    	function mysqli_select_db($mysql, $database):bool

\end{lstlisting}

A função da Listagem~\ref{lst:7} recebe como parâmetros:
\begin{itemize}
  \item \textbf{\$mysql}: \textit{connection string} de acesso à base de dados.
  \item \textbf{\$database}: nome da base de dados que deverá ser selecionada.
\end{itemize}

%-----

\subsubsection{\textit{mysql\_result vs nka\_mysqli\_result}}

Neste exemplo será apresentado nos excertos de código a alteração que teve de ser feita quando a função da biblioteca MySQL não foi continuada na biblioteca MySQLi, fazendo com que tivesse de ser criada uma equivalente.

A função \textit{mysql\_result} (listagem~\ref{lst:4}) obtém os dados de um determinado resultado.

\begin{lstlisting}[language={php},
                   caption={Função mysql\_result.},
                   label=lst:4]
    	function mysql_result($result, $row, $field=0):string

\end{lstlisting}

A função da Listagem~\ref{lst:4} recebe como argumentos:
\begin{itemize}
  \item \textbf{\$result}: resultado da \textit{query} \acrshort{sql} a ser executado.
  \item \textbf{\$row}: número da linha do resultado.
  \item \textbf{\$field}: nome do campo a ser procurado (tabela).
\end{itemize}


A função \textit{nka\_mysqli\_result} (listagem~\ref{lst:1}), utilizada para substituir a função \textit{deprecated mysql\_result} foi implementada recebendo os mesmos argumentos que a anterior, e retornando uma \textit{string}.

\begin{lstlisting}[language={php},
                   caption={Função para substituir mysql\_result.},
                   label=lst:1]
      if (!function_exists('nka_mysqli_result')) {
      	function nka_mysqli_result($res, $row, $field=0) {
      	  $res->data_seek($row);
      	  $datarow = $res->fetch_array();
      	  return $datarow[$field];
      	}
      }
\end{lstlisting}

A função da Listagem~\ref{lst:1} recebe como argumentos:
\begin{itemize}
  \item \textbf{\$result}: resultado da \textit{query} \acrshort{sql} a ser executado.
  \item \textbf{\$row}: número da linha do resultado.
  \item \textbf{\$field}: nome do campo a ser procurado (tabela).
\end{itemize}


\subsubsection{\textit{mysql\_field\_name vs mysqli\_field\_name}}

Nas Listagens~\ref{lst:5} e \ref{lst:2} está mais uma vez demonstrado o caso em que uma função da biblioteca MySQL é descontinuada para MySQLi e teve de se proceder à sua implementação.

A função \textit{mysql\_field\_name} (listagem~\ref{lst:5}) obtém o nome do campo especificado num resultado.


\begin{lstlisting}[language={php},
                   caption={Função mysql\_field\_name.},
                   label=lst:5]
  function mysql_field_name($result, $field_offset):string|false

\end{lstlisting}

A função da Listagem~\ref{lst:5} recebe como parâmetros:
\begin{itemize}
  \item \textbf{\$result}: resultado da \textit{query} \acrshort{sql} a ser executada.
  \item \textbf{\$field\_offset}: index/\textit{offset} de um campo numérico.
\end{itemize}


A função \textit{mysqli\_field\_name} (listagem~\ref{lst:2}), utilizada para substituir a função \textit{deprecated mysql\_field\_name} foi implementada no projeto recebendo os mesmos parâmetros que a anterior, e retornando uma \textit{string} ou \textit{false}.

\begin{lstlisting}[language={php},
                   caption={Função para substituir mysql\_field\_name.},
                   label=lst:2]
  if (!function_exists('mysqli_field_name')) {
    function mysqli_field_name($result, $field_offset){
$properties = mysqli_fetch_field_direct($result, $field_offset);
  return is_object($properties) ? $properties->name : false;
  }
  }
\end{lstlisting}

A função da Listagem~\ref{lst:2} recebe como parâmetros:
\begin{itemize}
  \item \textbf{\$result}: resultado da \textit{query} \acrshort{sql} a ser executada.
  \item \textbf{\$field\_offset}: index/\textit{offset} de um campo numérico.
\end{itemize}

%--

\subsubsection{\textit{Improvement} da função \textit{nka\_mysqli\_result}}

Neste caso está demonstrado um exemplo \textit{improvement} de uma função que já tinha sido desenvolvida para substituir uma função descontinuada da biblioteca MySQL.

A função \textit{nka\_mysqli\_result\_v2} (listagem~\ref{lst:3}), foi um \textit{improvement} da função previamente implementada \textit{nka\_mysqli\_result}. Esta alteração deveu-se ao facto de alguns erros na incompatibilidade das diferentes versões(7 e 8) do \acrshort{php} em que o \textit{\$datarow} chegava \textit{empty}, enquanto que na versão \acrshort{php}5.6 a mesma função não apresentava erro. Verificou-se que a mesma função se comportava de forma diferente de versão para versão e como tal, procedeu-se à sua remodelação passando agora a retornar \textit{string} ou \textit{null} resolvendo o problema.


\begin{lstlisting}[language={php},
                   caption={Improvement da função nka\_mysql\_result.},
                   label=lst:3]
      if (!function_exists('nka_mysqli_result_v2')) {
        function nka_mysqli_result_v2($res, $row, $field=0) {
      	  $res->data_seek($row);
      	  $datarow = $res->fetch_array();
      	  $retval = null;
      	  $cond = ! empty($datarow);
      	  if($cond){
      		    $retval = $datarow[$field];
      	  }
      	  return $retval;
      	}
      }
\end{lstlisting}

A função da Listagem~\ref{lst:3} recebe como argumentos:
\begin{itemize}
  \item \textbf{\$result}: resultado da \textit{query} \acrshort{sql} a ser executado.
  \item \textbf{\$row}: número da linha do resultado.
  \item \textbf{\$field}: nome do campo a ser procurado (tabela).
\end{itemize}


% - falar que no 5.6 nka_mysqli_result nao acusava erro mas em versoes superiores 7 e 8 dava erro. A mesma funçao comportava-se diferente nas 2 versões desenvolvidas.
