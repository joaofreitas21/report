\chapter{Conclusão Final}
\label{conc}
\section{Principais conclusões do estágio}

Neste período de quatro meses de estágio, o contacto com novas tecnologias, a integração numa empresa e a experiência de desenvolver projectos inseridos em ambientes de grande escala são algumas das competências adquiridas que completam a formação académica.

O estágio ofereceu-me uma experiência inicial muito importante no desenvolvimento de competências úteis para o meu futuro profissional, uma vez que desde muito cedo fui parte integrante em algumas reuniões de equipa de projetos, ou seja, experienciava de perto a maneira de como um produto era trabalhado e organizado até ao seu desenvolvimento e foram-me transmitidos muitos valores profissionais dentro da empresa.

Por fim, elaborando uma análise global ao trabalho desenvolvido é visível todo o trabalho realizado ao longo destes quatro meses. O objetivo, na sua totalidade, ainda se encontra um pouco longe do atingido, contudo o objetivo delineado para o estágio curricular foi cumprido uma vez que os requisitos impostos no ínicio do estágio foram concluídos, embora a última parte do desenvolvimento do módulo de apoio à gestão dos processos de formação - dossiers pedagógicos continue em fase de protótipo e como futura implementação no produto NkaAcademies.


%------ +1 paragrafo??!?!?!
\section{Pontos a melhorar/completar}

Neste projeto, os pontos a melhorar estão mais direcionados ao desenvolvimento da ferramenta de apoio ao preenchimento de documentos associados aos processos de formação, ou seja à segunda parte do estágio.

O módulo desenvolvido ainda está em fase protótipo e como potencial implementação futura no produto NkaAcademies, e sendo que existiram alguns requisitos estipulados, de menor prioridade de acordo com a metodologia \acrshort{moscow} que não foram implementados devido à falta de tempo considera-se que estes são alguns dos pontos que podem ser completados futuramente.
